% 使用书籍预定义模板
% 文章主要编码为 UTF-8, 编译器必须指定为 XeLaTeX
% 编译命令参见 build.sh
\documentclass[14pt]{ctexbook}

% 尺寸包
\usepackage{geometry}

% 页眉页脚
\usepackage{fancyhdr}

% 使用其中的 metalogo 显示引擎信息
% = fontspec, realscripts, metalogo
\usepackage{xltxtra}

% 必须支持 CJK. 自动设置中英文间距
% = l3kernel, l3packages, fontspec
\usepackage{xeCJK}

% 获取系统信息, 需要参数 -shell-escape
% 为 datetime2 提供 xelatex 不支持的秒
% = polyglossia, fontspec (if xelatex)
\usepackage{texosquery}
% 文档生成日期, 使用 iso 格式
% 如果使用 babel, 需要放在 babel 后面
\usepackage{datetime2}[
    showseconds=true, 
    showzone=true,
] 

% 文字装饰. 方框和颜色
\usepackage{tcolorbox}
% 各种链接
\usepackage{hyperref}

%%% 全局设置

% 页面参数. 装订线 8mm
\geometry{b6paper, includehead, bindingoffset=8mm,
    top=12mm, bottom=12mm, left=12mm, right=12mm}
% 段落距离
\setlength{\parskip}{0em}
% 段落缩进
\setlength{\parindent}{0em}
% 行距
\linespread{1.618}
% 超链接
\hypersetup{colorlinks,
    linkcolor={magenta!40!black},
    citecolor={cyan!40!black},
    urlcolor={blue!40!black},
}

% 各种字体
\setCJKmainfont{Noto Serif CJK SC}
\setCJKsansfont{Noto Sans CJK SC}
\setCJKmonofont{Noto Sans Mono CJK SC}
\setmainfont{Noto Serif CJK SC}
\setsansfont{Noto Sans CJK SC}
\setmonofont{Noto Sans Mono CJK SC}

\newCJKfontfamily\fontJPSansSec{Noto Sans CJK JP Black}
\newCJKfontfamily\fontJPSans{Noto Sans CJK JP}
\newCJKfontfamily\fontJPSerifItem{Noto Serif CJK JP Black}
\newCJKfontfamily\fontJPKlee{Klee One}
\newCJKfontfamily\fontSCSans{Noto Sans CJK SC}
\newCJKfontfamily\fontSCLXGW{LXGW WenKai Light}

%%% 脚手架

% 最后一次 commit 时间和版本
\newcommand{\gitversion}{\input{|"bash version.sh"}}
\newcommand{\engineversion}{
    {\XeLaTeX}
    {\the\eTeXversion\eTeXrevision-\the\XeTeXversion\XeTeXrevision}
}

% 词典章节 (字母)
\newcommand{\dictsection}[2]{
    \begin{tcolorbox}[halign=center, valign=center,
        colback=blue!5!white, colframe=blue!75!black,
        boxrule=0.7pt, arc=7pt
    ]
        \fontJPSansSec{\Huge #1 \Large #2} \label{sec:#1}
    \end{tcolorbox}
}

% 词典条目 4 个参数
\newcommand{\dictitem}[4]{
    % 用于索引的词条标签
    \label{item:#1}
    % 假名
    {\fontJPSerifItem #1}
    % 汉字表记
    \if\relax\detokenize{#2}\relax
      % 补一个全角空格
    \else
    \textcolor{blue!40!black}{\fontJPSans [#2]}
    \fi
    % 词条领域, 使用彩框包围
    \tcbox[on line, arc=2pt,
        colupper=red!40!black,colback=white,colframe=gray,
        boxsep=0pt,boxrule=0.7pt,left=1pt,right=1pt,top=1pt,bottom=1pt
    ]{\fontSCSans #3}
    % 词条解释
    #4
    \par
}

% 例句组 2 个参数
\newcommand{\itemquote}[2]{
    \tcbox[on line, arc=1pt,
    colupper=white,colback=gray,colframe=gray,
    boxsep=0pt,boxrule=0pt,left=1pt,right=1pt,top=1pt,bottom=1pt
    ]{\fontSCSans 例}
    \textcolor{blue!40!black}{\fontJPKlee 「#1」}
    {\fontSCLXGW #2}
}

% 词条链接
% 花括号为显示文字
% 方括号为词条标签, 留空则指向显示文字
\newcommand{\dictref}[2][]{
    \if\relax\detokenize{#1}\relax
    \hyperref[item:#2]{#2}
    \else
    \hyperref[item:#1]{#2}
    \fi
}