% 使用书籍预定义模板
% 文章主要编码为 UTF-8, 编译器必须指定为 XeLaTeX
% 编译命令参见 build.sh
\documentclass{book}
% 尺寸包
\usepackage{geometry}
% 动态行距
\usepackage{setspace}
% 页眉页脚
\usepackage{fancyhdr}
% 标题样式
\usepackage{titlesec}
% 目录样式
\usepackage{titletoc}
% 多栏
\usepackage{multicol}
% luatex 的 CJK 支持以及字体包. 自动设置中英文间距
\usepackage{luatexja}
\usepackage{luatexja-fontspec}
% 边注
\usepackage{marginnote}
% 文字装饰. 方框和颜色
\usepackage{tcolorbox}
% 各种链接
\usepackage{hyperref}

%%% 全局设置

% 页面参数. 装订线 8mm
\geometry{b6j, includehead, bindingoffset=8mm,
  marginparsep=3mm, marginparwidth=8mm,
top=12mm, bottom=12mm, left=12mm, right=12mm}
% 段落距离
\setlength{\parskip}{0em}
% 段落缩进
\setlength{\parindent}{0em}
% 超链接
\hypersetup{colorlinks,
  linkcolor={magenta!40!black},
  citecolor={cyan!40!black},
  urlcolor={blue!40!black},
}

% 各种字体
\setmainjfont{Noto Serif CJK SC}
\setsansjfont{Noto Sans CJK SC}
\setmonojfont{Noto Sans Mono CJK SC}

\newjfontfamily\fontJPSansSec{Noto Sans CJK JP Black}
\newjfontfamily\fontJPSans{Noto Sans CJK JP}
\newjfontfamily\fontJPSerifItem{Noto Serif CJK JP Black}
\newjfontfamily\fontJPSerif{Noto Serif CJK JP}
\newjfontfamily\fontJPKlee{Klee One}
\newjfontfamily\fontSCSans{Noto Sans CJK SC}
\newjfontfamily\fontSCSansSec{Noto Sans CJK SC Black}
\newjfontfamily\fontSCLXGW{LXGW WenKai Light}

% 页眉页脚
\setlength{\headheight}{14pt}

% 目录样式
\titleformat{\section}[runin]{\fontJPSansSec\Huge}{}{0em}{}{}
\AtBeginDocument{\addtocontents{toc}{\protect\thispagestyle{abstract}}}
\contentsmargin{0pt} % 目录页边距
\renewcommand{\thesection}{} % 章节号为空
\dottedcontents{section}[0em]{\fontJPSans}{0em}{0.5pc}

%%% 脚手架

% 最后一次 commit 时间和版本
\newcommand{\gitversion}{\input{|"bash version.sh"}}

% 词典章节 (字母)
\newcommand{\dictsection}[2]{
  \begin{tcolorbox}[halign=center, valign=center,
      colback=blue!5!white, colframe=blue!75!black,
      boxrule=0.7pt, arc=7pt
    ]
    \label{sec:#1}
    \section{#1}{\fontJPSansSec\Large #2}
  \end{tcolorbox}
}

% 词典条目领域, 使用彩框包围
\newcommand{\itemdomain}[1]{
  \tcbox[on line, arc=2pt,
    colupper=red!40!black,colback=white,colframe=gray,
    boxsep=0pt,boxrule=0.7pt,left=1pt,right=1pt,top=1pt,bottom=1pt
  ]{\fontSCSans #1}
}

% 词典条目 4 个参数
\newcommand{\dictitem}[4]{
  % 假名
  {\fontJPSerifItem #1}
  % 用于索引的词条标签, 放在主 key 后面, 防止链接错误定位
  \label{item:#1}
  % hyperref: 需要这东西才能跳转至词条, 否则只会跳转至章节
  \phantomsection
  % fancyhdr: 动态页眉标记, 同样放在主 key 后面, 防止页眉错误定位
  \markboth{#1}{#1}
  % 汉字表记
  \if\relax\detokenize{#2}\relax
  % \char"3000 % 补一个全角空格 (不甚美观, 暂且停用)
  \else
  \textcolor{blue!40!black}{\fontJPSans [#2]}
  \fi
  % 词条领域
  \itemdomain{#3}
  % 词条解释
  #4
  \par
}

% 例句组 2 个参数
\newcommand{\itemquote}[2]{
  \tcbox[on line, arc=1pt,
    colupper=white,colback=gray,colframe=gray,
    boxsep=0pt,boxrule=0pt,left=1pt,right=1pt,top=1pt,bottom=1pt
  ]{\fontSCSans 例}
  \textcolor{blue!40!black}{\fontJPKlee 「#1」}
  {\fontSCLXGW #2}
}

% 临时切换日文的正文字体
\newcommand{\ja}[1]{{\fontJPSerif #1}}
\newcommand{\jak}[1]{{\fontJPSerif 「#1」}}

% 词条链接
% 花括号为显示文字
% 方括号为词条标签, 留空则指向显示文字
\newcommand{\dictref}[2][]{%
  \if\relax\detokenize{#1}\relax
  \hyperref[item:#2]{#2}%
  \else
  \hyperref[item:#1]{#2}%
  \fi
}

% 原书页码
\newcounter{origpagenum}
\newcommand{\origpagenum}{%
\ifnum\value{origpagenum}<10 0\fi\arabic{origpagenum}
}
\newcommand{\origpagenumstyle}[1]{
  \tcbox[on line, arc=0pt,
    colupper=green!40!black, colback=green!5!white, colframe=green!40!black,
    boxsep=0pt, boxrule=0.5pt, left=1pt, right=1pt, top=1pt, bottom=1pt
  ]{\texttt{#1}}
}
\newcommand{\orignextpage}{
  \stepcounter{origpagenum}
  \marginnote[\origpagenumstyle{\origpagenum}]
  {\origpagenumstyle{\origpagenum}}
}

% 清除页眉页脚
\fancypagestyle{abstract}{
  \fancyhead{}
  \fancyfoot{}
  \renewcommand{\headrulewidth}{0pt}
}

% 正文页眉页脚
\fancypagestyle{body}{
  % 页码不重要, 放在不容易看到的中间
  \fancyhead[C]{\fontJPSansSec\thepage}
  % 左上角是首位词条
  \fancyhead[LE]{\fontJPSans\rightmark}
  % 右上角是末位词条
  \fancyhead[RO]{\fontJPSans\leftmark}
  \renewcommand{\headrulewidth}{0.5pt}
}
