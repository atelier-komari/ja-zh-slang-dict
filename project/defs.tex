% 使用书籍预定义模板
% 文章主要编码为 UTF-8, 编译器必须指定为 XeLaTeX
% 编译命令参见 build.sh
\documentclass[14pt]{book}

% 尺寸包
\usepackage{geometry}

% 页眉页脚
\usepackage{fancyhdr}

% luatex 的 CJK 支持以及字体包. 自动设置中英文间距
\usepackage{luatexja}
\usepackage{luatexja-fontspec}

% 获取系统信息, 需要参数 -shell-escape
% 为 datetime2 提供 xelatex 不支持的秒
% = polyglossia, fontspec (if xelatex)
\usepackage{texosquery}
% 文档生成日期, 使用 iso 格式
% 如果使用 babel, 需要放在 babel 后面
\usepackage{datetime2}[
	showseconds=true,
	showzone=true,
]

% 文字装饰. 方框和颜色
\usepackage{tcolorbox}
% 各种链接
\usepackage{hyperref}

%%% 全局设置

% 页面参数. 装订线 8mm
\geometry{b6paper, includehead, bindingoffset=8mm,
	top=12mm, bottom=12mm, left=12mm, right=12mm}
% 段落距离
\setlength{\parskip}{0em}
% 段落缩进
\setlength{\parindent}{0em}
% 行距
\linespread{1.618}
% 超链接
\hypersetup{colorlinks,
	linkcolor={magenta!40!black},
	citecolor={cyan!40!black},
	urlcolor={blue!40!black},
}

% 各种字体
\setmainjfont{Noto Serif CJK SC}
\setsansjfont{Noto Sans CJK SC}
\setmonojfont{Noto Sans Mono CJK SC}

\newjfontfamily\fontJPSansSec{Noto Sans CJK JP Black}
\newjfontfamily\fontJPSans{Noto Sans CJK JP}
\newjfontfamily\fontJPSerifItem{Noto Serif CJK JP Black}
\newjfontfamily\fontJPKlee{Klee One}
\newjfontfamily\fontSCSans{Noto Sans CJK SC}
\newjfontfamily\fontSCLXGW{LXGW WenKai Light}

%%% 脚手架

% 最后一次 commit 时间和版本
\newcommand{\gitversion}{\input{|"bash version.sh"}}

% 词典章节 (字母)
\newcommand{\dictsection}[2]{
	\begin{tcolorbox}[halign=center, valign=center,
			colback=blue!5!white, colframe=blue!75!black,
			boxrule=0.7pt, arc=7pt
		]
		\fontJPSansSec{\Huge #1 \Large #2} \label{sec:#1}
	\end{tcolorbox}
}

% 词典条目 4 个参数
\newcommand{\dictitem}[4]{
	% 用于索引的词条标签
	\label{item:#1}
	% 假名
	{\fontJPSerifItem #1}
	% 汉字表记
	\if\relax\detokenize{#2}\relax
		\char"3000 % 补一个全角空格
	\else
		\textcolor{blue!40!black}{\fontJPSans [#2]}
	\fi
	% 词条领域, 使用彩框包围
	\tcbox[on line, arc=2pt,
		colupper=red!40!black,colback=white,colframe=gray,
		boxsep=0pt,boxrule=0.7pt,left=1pt,right=1pt,top=1pt,bottom=1pt
	]{\fontSCSans #3}
	% 词条解释
	#4
	\par
}

% 例句组 2 个参数
\newcommand{\itemquote}[2]{
	\tcbox[on line, arc=1pt,
		colupper=white,colback=gray,colframe=gray,
		boxsep=0pt,boxrule=0pt,left=1pt,right=1pt,top=1pt,bottom=1pt
	]{\fontSCSans 例}
	\textcolor{blue!40!black}{\fontJPKlee 「#1」}
	{\fontSCLXGW #2}
}

% 词条链接
% 花括号为显示文字
% 方括号为词条标签, 留空则指向显示文字
\newcommand{\dictref}[2][]{
	\if\relax\detokenize{#1}\relax
		\hyperref[item:#2]{#2}
	\else
		\hyperref[item:#1]{#2}
	\fi
}