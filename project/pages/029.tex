\origpagenum{29}
